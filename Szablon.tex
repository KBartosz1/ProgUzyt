\documentclass[12pt, letterpaper, titlepage]{article}
\usepackage[left=3.5cm, right=2.5cm, top=2.5cm, bottom=2.5cm]{geometry}
\usepackage[MeX]{polski}
\usepackage[utf8]{inputenc}
\usepackage{graphicx}
\usepackage{enumerate}
\usepackage{amsmath} %pakiet matematyczny
\usepackage{amssymb} %pakiet dodatkowych symboli
\title{Pierwszy dokument LaTeX}
\author{Bartosz Kubeł}
\date{Październik 2022}
\begin{document}
\maketitle
\section{Opowieść} \begin{center}
Jej dom nie był miejscem dla tchórzy, dlatego prawdopodobnie nadal stał pusty. Kochała tak odważnie jak żyła i zasługiwała na to, by pokochano ją równie niepowstrzymanie. Działki na jego wybudowanie szukała jakiś czas. Stawała w centralnym punkcie terenu i sprawdzała czy to właśnie ten widok, na który chciałaby patrzeć przez resztę życia. W poważnych decyzjach nie była szybka, ani trochę, natomiast gdy już je podejmowała, była w nich do bólu stała. Znała dobrze siebie i swoje potrzeby, dzięki temu rzadko zdarzało jej się wahać. Dom postawiła ostatecznie na wzgórzu prawdy i faktycznie, miała z niego naprawdę wspaniały widok. Nie był zbyt duży, ponieważ nie chciała się w nim gubić ani szukać. Był optymalny. Taki w sam raz, by pomieścić wszystko, co było dla niej ważne. Drzwi do niego nie stały otworem, ale nie były też zamknięte na cztery spusty. Sąsiedzi mówili, że dłuższy czas wystawała na progu, tak jakby czekając na kogoś, ale ten ktoś nigdy nie nadjechał i pewnego dnia spojrzała w dal ostatni raz i od tamtej pory jej tam nie widzieli. Prawdopodobnie poszła robić swoje, co wnioskowali głównie po tym, że dwa razy w tygodniu biegała nago po ogrodzie z mieczem samurajskim w rękach (a przynajmniej robiła to zawsze wtedy, gdy przychodziła burza). Drzwi zostawiła uchylone na kilka milimetrów, czasami przebijało się przez nie światło. Chętni mogli zajrzeć do środka, bo nie postawiła przed żadnego muru ani płotu. Chętni mogli zajrzeć do środka, a reszta przebiegu wizyty zależała już w zasadzie tylko od ich manier, ale z tego, co mówili, większość rezygnowała tuż przed drzwiami i co ciekawe, tylko po to, by później cyklicznie wracać i wystawać jej pod drzwiami, patrząc martwo w okna. W środku było dużo jasnej przestrzeni. W zasadzie jej wnętrze odzwierciedlało wnętrze. Raczej nie gromadziła rzeczy, ceniła sobie ludzi, czas i zdarzenia. Nie była pedantką, ale lubiła porządek i przejrzyste sytuacje. Nie kolekcjonowała pierdół ani prawdopodobnych przydasiów na zaś. O niezbędne bardzo dbała, zbędne zaś wylatywało przez drzwi szybciej, niż się pojawiło. Jeśli już wejdziesz do środka i akurat dobrze trafisz, możesz zastać ją tańczącą w salonie. Bądź wtedy cierpliwy, bo z pewnością Cię wtedy nie zauważy. W kuchni, jak zawsze, będzie stała czysta wódka, więc zrób sobie drinka. Usiądź na kanapie lub dołącz do niej. Dołączysz? Dołącz. To kobieta, która, jak mało co, ceni sobie odwagę. Nie rezygnuj. To kobieta, która nie bawi się w konkursy. Nie poddawaj się. To kobieta, która pomoże Ci dostrzec w Tobie wszystko to, czego sam nie widzisz. Uśmiechnij się. To kobieta, której nic więcej nie potrzeba.
\end{center}

\begin{flushright}
\textbf{Marta Kostrzyńska}
\end{flushright}
\newpage
\section{Przepisy}
\subsection{Zupy}
\subsubsection{Jesienna zupa z dyni}
\begin{flushleft}
\textbf{Składniki:}\\-Dynia np.Hakkaido (1kg.)\\-Rosół z kury Knorr (1szt.)\\-Mleko Kokosowe (200mililitrów)\\-Cebula (1szt.)\\-Ząbek czosnku (1szt.)\\-Kawałek świeżego imbiru (ok. 5cm)\\-Brązowy cukier (1 łyżka)\\-Pestki z dyni (50 gram)\\-Olej rzepakowy (1 łyżka)\\
\end{flushleft}
\begin{center}
\textbf{Przygotowanie Krok po Kroku}
\end{center}
\begin{flushleft}
\textbf{Krok 1}
Obierz i pokrój dynię na kawałki. Posiekaj cebulę i czosnek. Obierz i zetrzyj na tarce imbir. Przesmaż cebulę i czosnek na oleju.\\
\end{flushleft}

\begin{flushleft}
\textbf{Krok 2}
Dodaj dynię, imbir i cukier. Smaż przez 5 min razem. Dodaj 1 l wody i kostki bulionowe Knorr. Całość gotuj do miękkości, ok. 10 min.\\
\end{flushleft}

\begin{flushleft}
\textbf{Krok 3}
Zmiksuj zupę na gładko. Dodaj mleko kokosowe i jeszcze raz zagotuj. Podawaj z pestkami dyni.\\
\end{flushleft}
\begin{center}
\textbf{Dobra rada:}\\
\end{center}
Pestki z dyni będą jeszcze lepsze gdy uprażycie je na suchej patelni przez 5 min. Pamiętajcie o mieszaniu by uniknąć przypalenia. Zamiast pestek dyni możesz też dodać uprażone pestki słonecznika. Porcja dania dostarcza istotną ilość witaminy A potrzebnej w codziennej diecie. Witamina A pomaga w prawidłowym funkcjonowaniu układu odpornościowego. Spożywaj, jako element zrównoważonej i zróżnicowanej diety oraz zdrowego trybu życia.
\newpage
\subsubsection{Barszcz biały Wielkanocny}
\begin{flushleft}
\textbf{Składniki:}\\-Kiełbasa biała surowa (4 szt.)\\-Włoszczyzna (1 pęczek)\\-Barszcz biały Knorr (1 opakowanie)\\-Chran ze słoika (1 łyżka)\\-Jajka (4 szt.)\\-Ziemniaki (5 szt.)\\-Śmietana Kremowa (100 mililitrów)\\-Pieprz czarny mielony)\\-Majeranek (1 łyżka)\\
\end{flushleft}
\begin{center}
\textbf{Przygotowanie Krok po Kroku}
\end{center}
\begin{flushleft}
\textbf{Krok 1}
Gotuj wywar na włoszczyźnie razem z pokrojonymi w kostkę ziemniakami, aż ziemniaki będą gotowe, po czym usuń włoszczyznę. Kiełbasę sparz przez minutę.
\end{flushleft}
\begin{flushleft}
\textbf{Krok 2}
Po wyciągnięciu z rosołu, kiełbasę pokrój w plastry i dodaj z powrotem do zupy.
\end{flushleft}
\begin{flushleft}
\textbf{Krok 3}
Dodaj do garnka zawartość opakowania Barszcz Biały Knorr wraz z chrzanem i gotuj kolejne 10 minut.
\end{flushleft}
\begin{flushleft}
\textbf{Krok 4}
Na koniec gotowania dodaj pieprz, majeranek i śmietanę. Podawaj żurek z jajkiem rozkrojonym w ćwiartkę.
\end{flushleft}
\newpage
\subsubsection{Rosół z kury}
\textbf{Składniki:}\\-Kura (około 1kg)\\-Rosół z kury Knorr (2szt.)\\-Makaron (200gram)\\-Marchewka (2szt.)\\-Cebula (1szt.)\\-Korzeń Pietruszki (1szt.)\\-Natka Pietruszki (1 pęczek)\\-Mały seler (1szt.)\\-Por (1szt.)\\-Liść laurowy (2szt.)\\-Ziele angielski (2szt.)
\begin{flushleft}
\begin{center}
\textbf{Przygotowanie Krok po Kroku}
\end{center}
\textbf{Krok 1}
Mięso umyj, następnie zalej dwoma litrami zimnej wody.
\end{flushleft}
\begin{flushleft}
\textbf{Krok 2}
Powoli doprowadź do wrzenia. Zbierz powstałą pianę. Warzywa obierz.
\end{flushleft}
\begin{flushleft}
\textbf{Krok 3}
Cebulę przekrój na pół. Opal nad gazem lub przypal na patelni.
\end{flushleft}
\begin{flushleft}
\textbf{Krok 4}
Do wywaru dodaj warzywa, przyprawy oraz dwie kostki Rosołu z kury Knorr. Gotuj na wolnym ogniu przez około 3 godziny.
\end{flushleft}
\begin{flushleft}
\textbf{Krok 5}
Po ugotowaniu odstaw rosół na 20 minut, następnie powoli przecedź przez gęste sito lub przez gazę. Rosół powinien być klarowny.
\end{flushleft}
\begin{flushleft}
\textbf{Krok 6}
Ugotowane warzywa pokrój w drobną kostkę, mięso z kury w drobne kawałki. Natkę pietruszki posiekaj. Rosół podawaj z ugotowanym wcześniej makaronem, warzywami i mięsem, posypany natką pietruszki. W razie potrzeby możesz doprawić rosół solą i pieprzem.
\end{flushleft}
\begin{center}
\textbf{Dobra Rada}
\end{center}
Aby natka z pietruszki dłużej zachowała świeżość, opłucz ją, ułóż na ręczniku papierowym, włóż do woreczka foliowego i umieść w lodówce. Po przygotowaniu tego dania możesz zrobić jeszcze Sałatkę jarzynową. \underline{ZERO WASTE} w kuchni.
\newpage
\subsection{Dania Główne}
\subsubsection{Kolorowe risotto z kurczakiem}
\textbf{Składniki:}\\-Filety z piersi z kurczaka (200 gram)\\-Gotowa mrożonka mieszanka warzywna (400 gram)\\-Bulionetka drobiowa Knorr (1 szt.)\\-Przyprawa do kurczaka Knorr (1 szt.)\\-Biała cebula (1 szt.)\\-Ryż arborio (200 gram)\\-Zioła Prowansalskie\\-Starty żółty ser (3 łyżki)\\-Śmietana (4 łyżki)\\-Oliwa (1 łyżka)\\-Masło (20 gram)\\-Woda (550 mililitrów)\\
\begin{center}
\textbf{Przygotowanie Krok po Kroku}
\end{center}

\begin{flushleft}
\textbf{Krok 1}
Cebulę drobno pokrój i podsmaż ją w szerokim rondlu na rozgrzanej oliwie wraz z surowym ryżem, przez 1 minutę.
\end{flushleft}
\begin{flushleft}
\textbf{Krok 2}
Pierś z kurczaka pokrój w kostkę, oprósz Przyprawą do kurczaka Knorr i obsmaż na patelni.
\end{flushleft}
\begin{flushleft}
\textbf{Krok 3}
Do podsmażonego ryżu wlej 300 mililitrów wody oraz jedną bulionetkę Knorr. Duś całość około 12 minut.
\end{flushleft}
\begin{flushleft}
\textbf{Krok 4}
Następnie dodaj mieszankę warzyw, kurczaka, zioła prowansalskie oraz 250 mililitrów wody i duś kolejne 6 minut.
\end{flushleft}
\begin{flushleft}
\textbf{Krok 5}
Pod koniec duszenia dodaj starty żółty ser, śmietanę oraz masło. Energicznie mieszaj i odstaw z ognia na kilka minut. Natychmiast podawaj.

\end{flushleft}
\begin{center}
\textbf{Dobra Rada}
\end{center}
Kolorowe risotto z kurczakiem i warzywami można przygotować w wersji wege, wystarczy dodać tofu zamiast kurczaka. 
\end{document}
